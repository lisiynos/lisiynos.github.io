\documentclass[11pt,a4paper,oneside]{article}
\usepackage[utf8]{inputenc}
\usepackage[english,russian]{babel}

\begin{document}

1. Постановка задачи

Рассмотрим следующую задачу. Дано $n$ ящиков, пронумерованных числами от $1$ 
до $n$, в каждом из которых лежит несколько шариков. Известно, что $n$ 
достаточно велико. Нам нужно уметь быстро выполнять следующие операции: 

1) $add(c, w)$ - изменить количество шариков в ящике $c$, прибавив к нему $w$ шариков 
($w$ может быть отрицательным); 

2) $sum(a, b)$ - посчитать количество шариков в нескольких ящиках, расположенных по порядку с номерами от $a$ 
до $b$, то есть на отрезке $[a,b]$.

Мы можем просто завести массив, в котором будем хранить число шариков в ящиках. 
Почему эта идея может оказаться неоптимальной? 
При изменении числа шариков в ящике мы меняем значение элемента массива с номером $c$, что выполняется 
за $O(1)$. Но чтобы посчитать сумму в нескольких идущих подряд ящиках, нам нужно будет сложить 
$b - a + 1$ чисел. Поскольку длина отрезка может быть почти $n$, то выполняется порядка $n$ сложений, то есть
операция {\it sum} выполняется за $O(n)$. При достаточно большом числе запросов суммы на отрезке решение 
задачи перебором является далеко не лучшим.

2. Корневая оптимизация

рисунок 1

Первая идея, которая обычно приходит в голову, это разделить $n$ ящиков на $x$ 
групп таким образом, что первые несколько ящиков по порядку попадают в первую 
группу, следующие - во вторую, и так далее. Далее заведем дополнительный массив 
из $x$ элементов, где будет храниться количество шариков в каждой группе. Тогда 
при выполнении операции {\it add} нам нужно будет выполнить два действия: 
изменить значение в соответствующей группе и в самом элементе. Зато операция 
{\it sum} потребует меньших затрат: нужно сложить элементы группы, которой 
принадлежит число $a$ (обозначим ее $A$), начиная с элемента с номером $a$ и 
кончая последним, элементы группы, которой принадлежит число $b$ (обозначим ее 
$B$), начиная с первого и кончая элементом с номером $b$, что требует временных 
затрат порядка $O(\frac{n}{x})$ и прибавить к ним суммы элементов в группах, 
начиная с группы, следующей за $A$ и заканчивая группой, расположенной перед 
группой $B$, что требует временных затрат порядка $O(x)$. Всего нужно выполнить 
примерно $\frac{n}{x} + x$ операций. Если мы подберем число $x$ оптимальным 
образом, то эта идея называется корневой оптимизацией. Чтобы ответить на вопрос, 
какое $x$ является наилучшим, нужно устремить к минимуму выражение 
$\frac{n}{x} + x$. Возьмем от него производную по $x$ и приравняем ее к нулю: 
$-\frac{n}{x^{2}}+1=0$. Отсюда получаем $x=\sqrt{n}$. Таким образом, 
оптимальное количество групп, на которые нужно разбить ящики: $[\sqrt{n}]$. 
Иы добились оценки времени выполнения операции {\it sum} $O(\sqrt{n})$, правда, 
увеличился расход по памяти, но всего лищь на $\sqrt{n}$ - число дополнительных
 ящиков для хранения сумм элементов в группах.

3. Дерево отрезков

Следующая мысль, которая приходит в голову, это то, что мы можем завести не 
один вспомогательный уровень, а много. Каждый уровень будет служить для 
уменьшения вычислений на уровне ниже. Для простоты реализации проще иметь 
одинаковое число возможных переходов на каждом уровне. Таким образом, мы 
приходим к дереву, из верхнего узла которого $x$ переходов, из каждого из 
получившихся $x$ узлов также $x$ переходов, и так далее. На самом нижнем 
уровне получается $x^{h}$ узлов, где $h$ - высота дерева, что должно быть 
равно $n$. Отсюда $h = \log_{x}n$.

рисунок 2

Дальше нам нужно выяснить, какое $x$ является оптимальным. Теперь при 
выполнении операции {\it sum}, так же как и {\it add}, нужно выполнить 
$hx$ действий. Нам нужно минимизировать это выражение: $h * x\to\min$. 
Возьмем от него производную и приравняем ее к нулю: 
$(\log_{x}n * x)' = 0$. Получаем $x = e$. Таким образом, из каждого узла 
должно выходить $e$ веток. Ближайшее к $e$ целое число - это $3$. Однако, 
учитывая, что операции умножения и деления на $2$ на процессорах с 
двоичной арифметикой выполняются значительно быстрее, в качестве $x$ 
берут $2$, а не $3$.

3.1. Удобный способ хранения дерева отрезков

Обозначим за $N$ ближайшее сверху к $n$ целое число такое, что $N$ является степенью двойки. Тогда $N = 2^{h}$. Элементы массива, начиная с $1$ по $n$ образуют начало нижнего уровня дерева отрезков, оставшиеся нижние узлы дерева от $n + 1$ до $N$ заполняются нулями. 

Перенумеруем узлы дерева так, как показано на рисунке и разместим 
значения, хранящиеся в узлах дерева, в массиве в соответствии с этой 
нумерацией. Сначала расположен самый верхний узел дерева, потом узлы
второго сверху уровня слева направо, потом третьего слева направо, и 
так далее. В конце идут значения, данные изначально. 

рисунок 3

Тогда переход сверху вниз выполняется по следующей схеме:

рисунок 4

, а снизу вверх так:

рисунок 5

3.2. Сколько места занимает дерево отрезков?

Кроме массива длины $N$, нужно еще $N - 1$ дополнительных 
элементов, всего получается $2N - 1$. Причем удобно нумерацию 
начинать с $1$, чтобы проще 
вычислять переходы по приведенным схемам. Тогда в языках, где 
массив начинается с нуля (например, C++), пропадает нулевой элемент, 
затраты по памяти составляют $2N$.


3.3. Реализация

При реализации дерева отрезков можно написать операции add и sum 
двумя способами: циклом и рекурсией. Кроме того, можно идти по 
дереву отрезков вниз, а можно вверх. Итого, для каждой операции 
четыре возможных варианта.

3.3.1. Операция {\it add}

Чтобы выполнить операцию {\it add}, нужно просто пройти по дереву вверх 
как показано на рис. 5, увеличивая значение в каждом 
встреченном узле на $w$. 

Реализация на C++ операции {\it add} циклом, по дереву идем вверх

файл 1.cpp

3.3.2. Операция {\it sum}

Заведем переменную $res$, в которой будем хранить текущее 
значение суммы. Изначально стоим на нижнем уровне дерева,
нужно посчитать сумму элементов от $a$ до $b$ включительно.
Заметим, что для левой границы $a$, когда выполняем переход вверх,
нужно прибавлять значение в узле дерева к $res$, когда $a$ - 
нечетное, после чего левую границу нужно сдвинуть на $1$ вправо.
Для правой границы, соответственно, наоборот. На каком-то шаге
левая граница становится больше правой и цикл заканчивается.
 
Реализация на C++ операции {\it sum} циклом, по дереву идем вверх

файл 2.cpp

Реализация на C++ операции {\it add} рекурсией, по дереву идем ?

файл 3.cpp

Реализация на C++ операции {\it sum} рекурсией, по дереву идем ?

файл 4.cpp  

3.4.

Если мы задумаемся над следующим вопросом: при каких n нам действительно выгодно применить 
дерево отрезков, а когда не стоит тратить время и усилия на его написание, когда лучше написать 
корневую оптимизацию или даже может оказаться, что оно работает медленнее перебора.
С этой целью можно предложить такой эксперимент:
написать дерево отрезков, корневую оптимизацию и перебор и запустить все три алгоритма при некоторых n, например, при 
$n = 5, 10, 10^2, 10^3, 10^4,\ldots, 10^{12}$, записывая время работы каждого в два массива, а потом сравнить, 
что получилось.

3.5. Стандартные ошибки при написании дерева отрезков:

1) Обычно, когда рисуют дерево отрезков, никто не забывает, что нужно увеличить $n$ до 
ближайшей степени двойки, но во время реализации оставляют $n$ как есть, тесты из примеров 
такой код проходит, но на больших тестах получается неверный ответ.

2) Перед выполнением операции add или sum нужно не забыть прибавить 
к номеру элемента в исходном массиве 
$N-1$, чтобы получить номер элемента в дереве.

%4. Дерево отрезков с добавлением на отрезке







\end{document}
 